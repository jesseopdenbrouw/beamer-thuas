
\documentclass[fleqn,aspectratio=43]{beamer}
\usepackage[utf8]{inputenc}
\usepackage[T1]{fontenc}

%\usepackage{showframe}
\usetheme{thuas}


\title{Digitale Systemen}
\author{Jesse op den Brouw}
\date{VAK/2021-2022}


\begin{document}

\begin{frame}
\titlepage
\end{frame}

\begin{frame} 
\frametitle{There Is No Largest Prime Number} 
%\framesubtitle{The proof uses \textit{reductio ad absurdum}.} 
\begin{theorem}
There is no largest prime number. \end{theorem} 
\begin{enumerate} 
\item<1-| alert@1> Suppose $p$ were the largest prime number. 
\item<2-| alert@2> Let $q$ be the product of the first $p$ numbers. 
\item<3-| alert@3> Then $q+1$ is not divisible by any of them. 
\item<1-> But $q + 1$ is greater than $1$, thus divisible by some prime
number not in the first $p$ numbers.
\end{enumerate}
\end{frame}

\begin{frame}
\frametitle{Formula}
\begin{itemize}
\item The formula is:
\begin{equation*}
\left[F(x)\right]^b_a = \int_a^b x^2 + 2x + 1 \, \mathsf{d} x
\end{equation*}
\begin{equation*}
\zeta (s) = \sum_{n=1}^\infty \dfrac{1}{n^{\;\!s}}
\end{equation*}
\end{itemize}
\end{frame}

\begin{frame}{Itemize}
En nu wat tekst.
\begin{itemize}
\item one
\begin{itemize}
\item one
\begin{itemize}
\item one
\end{itemize}
\end{itemize}
\item two
\end{itemize}
En weer wat tekst
\end{frame}


\begin{frame}{Enumerate}
Tekst
\begin{enumerate}
\item een
\item twee
\begin{enumerate}
\item een
\item twee
\end{enumerate}
\end{enumerate}
\end{frame}



\end{document}
