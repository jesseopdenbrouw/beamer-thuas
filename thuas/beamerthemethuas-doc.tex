%%
%%
%% thuasexample.tex
%%
%% Definitions for the beamer layout for
%%
%% The Hague University of Applied Sciences (THUAS)
%%
%% (c) J. op den Brouw  <J.E.J.opdenBrouw@hhs.nl>
%%
%%
%% This work may be distributed and/or modified under the
%% conditions of the LaTeX Project Public License, either version 1.3
%% of this license or (at your option) any later version.
%% The latest version of this license is in
%%   http://www.latex-project.org/lppl.txt
%% and version 1.3 or later is part of all distributions of LaTeX
%% version 2005/12/01 or later.
%%
%% This work has the LPPL maintenance status `maintained'.
%% 
%% The Current Maintainer of this work is J.E.J. op den Brouw
%%


%% Use ``dutch'' as class option
\documentclass[fleqn,aspectratio=169,dutch]{beamer}

%% Babel is needed for correct translations
%% It needs the ``dutch'' option
\usepackage[dutch]{babel}


%% Show frames, for testing
%\usepackage{showframe}

%% The THUAS beamer style
\usetheme{thuas}

%% Find out what engine we're running...
\usepackage{iftex}

%% Settings for fonts et al.
\ifnum 0\ifxetex 1\fi\ifluatex 1\fi>0
\usepackage[math-style=TeX]{unicode-math}
%\usepackage{unicode-math}
\usepackage{fontspec}
\setmainfont[Ligatures=TeX]{Calibri}
\defaultfontfeatures{Scale=MatchUppercase}
\setsansfont{Calibri}
\setmonofont{Consolas}
\setmathfont[slash-delimiter=frac]{Cambria Math}
%% Next messes up the math font, try it, but do not use it!
%\setmathfont[range=up]{Calibri}
%\setmathfont[range=it]{Calibri Italic}
%\setmathfont[range=bfup]{Calibri Bold}
%\setmathfont[range=bfit]{Calibri Bold Italic}
\setoperatorfont\normalfont % For log, sin, cos, etc.
\else
%%% Set input encoding to UTF-8
\usepackage[utf8]{inputenc}
%% T1 encoding
\usepackage[T1]{fontenc}
%% Default to helvet
%\usepackage{helvet}
%% Math package??

%% Use the Nimbus Mono fonts
\usepackage{nimbusmono}
\fi

%% Use microtype
%\usepackage[stretch=10]{microtype}


\usepackage{listings}
\lstset{
    language=C,
    basicstyle=\ttfamily\small,
    xleftmargin=1em
}

\usepackage{booktabs}

%%
%%
%%
\title{THUAS Beamer Slides}
\subtitle{Subtitel wordt niet weergegeven}
\author{Jesse op den Brouw}
\date{\today}


\begin{document}


%% Title page
\maketitle

\begin{frame}[fragile]{THUAS Beamer Slides}
\begin{itemize}
\item Dit is de onofficiële realisatie van slides met het THUAS-thema
\item Eerst gebruik je de beamer-class: \lstinline|\documentclass{beamer}|
\item Daarna laadt je de theme met: \lstinline|\usetheme{thuas}|
\item Het werkt met \LaTeX, Xe\LaTeX en Lua\LaTeX
\item Voorlopig alleen in het Nederlands
\item De titelpagina is niet conform de regels, maar komt in de buurt
\item De navigatie-buttons werken nu nog niet
\end{itemize}
\end{frame}

\begin{frame}[fragile]{THUAS Beamer Slides}
\begin{itemize}
\item De officiële realisatie is met aspect ratio 16:9, dus:
\begin{itemize}
\item Gebruik \lstinline|\documentclass[aspectratio=169]{beamer}|
\end{itemize}
\item Maar veel beamers werken nog met 4:3, dus:
\begin{itemize}
\item Gebruik \lstinline|\documentclass[aspectratio=43]{beamer}|
\end{itemize}
\item Er zijn nog andere formaten maar die worden niet ondersteund
\item Er zijn verschillen in de titelpagina tussen 16:9 en 4:3
\begin{itemize}
\item Dat komt o.a.\@ door het plaatsen van het plaatje op de titelpagina
\end{itemize}
\item Er zijn verschillen tussen Xe-, Lua- en pdf\LaTeX
\begin{itemize}
\item Dat komt o.a.\@ door de vorm en de grootte van de gebruikte fonts
\end{itemize}
\end{itemize}
\end{frame}

\begin{frame}[fragile]{THUAS Beamer Slides}
Om correct gebruik te maken van het Nederlands, gebruik
\begin{lstlisting}
\documentclass[dutch]{beamer}
\end{lstlisting}
en
\begin{lstlisting}
\usepackage[dutch]{babel}
\end{lstlisting}
Dan worden environments als \lstinline|theorem| en \lstinline|proof| van de correcte namen voorzien
\end{frame}

\begin{frame}[fragile]{THUAS Beamer Slides}
\begin{itemize}
\item Het standaard lettertype in pdf\LaTeX{} is helvet
\item Dit kan je uitzetten met \lstinline|\usetheme[nohelvet]{thuas}|
\item Gebruik voor Xe\LaTeX{} en Lua\LaTeX{} \lstinline|\setmainfont| etc.
\end{itemize}
\end{frame}

\begin{frame}[fragile]{THUAS Beamer Slides}
\begin{itemize}
\item Subtitel op titelpagina wordt \emph{niet} weergegeven
\begin{itemize}
\item Deze subtitel wordt gewoon genegeerd
\end{itemize}
\item Subtitels op frames worden \emph{niet} weergegeven
\begin{itemize}
\item Deze subtitels worden gewoon genegeerd
\end{itemize}
\item De inhoud van een slide wordt \emph{niet} gecentreerd
\begin{itemize}
\item De huisstijl is zo
\end{itemize}
\item Wil je toch gecentreerde slides, gebruik dan \lstinline|\usetheme[c]{thuas}|
\end{itemize}
\end{frame}

\begin{frame}
\frametitle{Formules kunnen ook}
\begin{itemize}
\item The formules zijn:
\begin{align*}
\left|F(x)\right|^b_a &= \int_a^b x^2 + 2x + 1 \, \mathsf{d} x \\
\zeta (s) &= \sum_{n=1}^\infty \dfrac{1}{n^{\;\!s}} \\
M&\approx\frac{\pi}{4}\left(\frac{2d}{\lambda_o}\right)^2\left(\mathrm{NA}\right)^2
\end{align*}
\end{itemize}
\end{frame}

\begin{frame}{Formules kunnen ook}
Nu zonder itemize
\begin{equation*}
O=\pi r^2
\end{equation*}
\begin{multline*}  
K=\displaystyle{\frac{1}{2}m_1 L_1^2 \dot{\theta_1}^2+\frac{1}{2} m_2[L_1^2 \dot{\theta_1}^2+L_2^2 \dot{\theta_2}^2+2 L_1 L_2 \dot{\theta_1}\dot{\theta_2}\cos(\theta_1-\theta_2)]} \\
 \displaystyle{+\frac{1}{2}m_3[L_1^2 \dot{\theta_1}^2+L_2^2 \dot{\theta_2}^2+L_3^2+ \dot{\theta_3}^2+2 L_1 L_2 \dot{\theta_1}\dot{\theta_2}\cos(\theta_1-\theta_2)}
\end{multline*}
\begin{equation*}
\mathrm{e}^{\, \mathrm{j}\alpha} = \cos \alpha + \mathrm{j} \sin \alpha
\end{equation*}
\end{frame}

\begin{frame}
\frametitle{Voorbeeld met een itemize}
Voorbeeld met een \texttt{itemize} en gaat tot drie niveaus diep.
\begin{itemize}
\item item
\begin{itemize}
\item sub item
\begin{itemize}
\item sub sub item
\end{itemize}
\end{itemize}
\item item
\end{itemize}
En weer wat tekst
\end{frame}

\begin{frame}{Voorbeeld met een enumerate}
Voorbeeld met een \texttt{enumerate} en gaat tot drie niveaus diep.
\begin{enumerate}
\item een
\item twee
\begin{enumerate}
\item een
\item twee
\begin{enumerate}
\item een
\item twee
\end{enumerate}
\end{enumerate}
\item drie
\end{enumerate}
\end{frame}


\begin{frame}{Voorbeeld van een description}
De label wordt vet en rechts uitgelijnd afgedrukt

De label kan ongeveer 9 karakters bevatten

Daarna wordt er ingesprongen
\begin{description}
\item[123456789] Dit is een hele lange tekst en ik denk dat deze zin over twee regels verspreid zal zijn
\item[label] description
\end{description}
\end{frame}

%% A frame with a listing NEEDS fragile
\begin{frame}[fragile]{Een frame met code}
Een frame met daarin code met \texttt{lstlistings} moet getypeerd worden met \texttt{fragile}, anders werkt het niet

\begin{lstlisting}
\begin{frame}[fragile]{Titel}
\end{lstlisting}

De frame wordt in een bestand \lstinline|\jobname.vrb| geplaatst en daarna ingelezen

\begin{lstlisting}
#include <stdio.h>

int main(void) {
    printf("Thuas Beamer Slides!\n");
}
\end{lstlisting}

Niet getest met \lstinline|minted|
\end{frame}

\begin{frame}[fragile]{Plaatjes en voetnoot}
Een voetnoot\footnote{Dit is een voetnoot}

Een voetnoot\footnote{Dit is een voetnoot}

Een plaatje (met een \lstinline|\fbox| en \lstinline|\fboxsep=0pt|)

\begin{figure}[!ht]
\fboxsep=0pt
\fbox{\includegraphics[scale=1]{thuaslogo}}
\caption{Dit is een plaatje}
\end{figure}

\end{frame}

\begin{frame}{Tabellen}
Een tabel kan ook (deze is met \texttt{booktabs})

\begin{table}[!ht]
\caption{Een tabel}
\begin{tabular}{ll}
\toprule
Iets & En nog iets\\
\midrule
Aaa & BBB\\
\bottomrule
\end{tabular}
\end{table}
\end{frame}

\begin{frame}[fragile]{Gory details...}
The theme package bestaat uit
\begin{itemize}
\item \texttt{beamerthemethuas.sty}
\begin{itemize}
\item Deze moet je aanroepen met \lstinline|\usetheme{thuas}|
\end{itemize}
\item \texttt{beamercolorthemethuas.sty}
\begin{itemize}
\item Hierin zijn de kleuren gedefinieerd
\end{itemize}
\item \texttt{beamerinnerthemethuas.sty}
\begin{itemize}
\item Hierin is de opmaak van de inhoud gedefinieerd (ook de titelpagina)
\end{itemize}
\item \texttt{beamerouterthemethuas.sty}
\begin{itemize}
\item Hierin is de opmaak \emph{rond} de inhoud gedefinieerd (header, footer)
\end{itemize}
\end{itemize}
\end{frame}

\begin{frame} 
\frametitle{There Is No Largest Prime Number} 
\begin{theorem}
There is no largest prime number.
\end{theorem} 
\begin{proof}
\begin{enumerate} 
\item<1-| alert@1> Suppose $p$ were the largest prime number. 
\item<2-> Let $q$ be the product of the first $p$ numbers. 
\item<3-> Then $q+1$ is not divisible by any of them. 
\item<1-> But $q + 1$ is greater than $1$, thus divisible by some prime
number not in the first $p$ numbers.
\end{enumerate}
\end{proof}
\end{frame}


%%% Test slide van Jesse
%\newdimen\efkes\efkes=12.80cm
%\makeatletter
%\begin{frame}{test slide van Jesse}
%
%%% Test if babel loaded
%\ifdefined\bbl@loaded
%\bbl@loaded
%\else
%Babel not loaded
%\fi
%
%%% translater always loaded by beamer
%\trans@languages
%
%%% 12.80cm in points
%\the\efkes
%
%\end{frame}
%\makeatother

\end{document}
