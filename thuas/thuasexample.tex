%%
%%
%% thuasexample.tex
%%
%% Definitions for the beamer layout for
%%
%% The Hague University of Applied Sciences (THUAS)
%%
%% (c) J. op den Brouw  <J.E.J.opdenBrouw@hhs.nl>
%%
%%
%% This work may be distributed and/or modified under the
%% conditions of the LaTeX Project Public License, either version 1.3
%% of this license or (at your option) any later version.
%% The latest version of this license is in
%%   http://www.latex-project.org/lppl.txt
%% and version 1.3 or later is part of all distributions of LaTeX
%% version 2005/12/01 or later.
%%
%% This work has the LPPL maintenance status `maintained'.
%% 
%% The Current Maintainer of this work is J.E.J. op den Brouw
%%

\documentclass[fleqn,aspectratio=43]{beamer}

% Is needed to load fonts
\usepackage[T1]{fontenc}

%% Show frames, for testing
%\usepackage{showframe}

%% The THUAS beamer style
\usetheme{thuas}

%% Find out what engine we're running...
\usepackage{iftex}

%% Settings for fonts et al.
\ifnum 0\ifxetex 1\fi\ifluatex 1\fi>0
\usepackage[math-style=TeX]{unicode-math}
%\usepackage{unicode-math}
\usepackage{fontspec}
\setmainfont[Ligatures=TeX]{Calibri}
\defaultfontfeatures{Scale=MatchUppercase}
\setsansfont{Calibri}
\setmonofont{Consolas}
\setmathfont[slash-delimiter=frac]{Cambria Math}
\setmathfont[range=up]{Calibri}
\setmathfont[range=it]{Calibri Italic}
\setmathfont[range=bfup]{Calibri Bold}
\setmathfont[range=bfit]{Calibri Bold Italic}
\setoperatorfont\normalfont % For log, sin, cos, etc.
\else
%% Default to helvet
%\usepackage{helvet}
%% Use the Charter and Nimbus Mono fonts
\usepackage{nimbusmono}
%%% Set input encoding to UTF-8
\usepackage[utf8]{inputenc}
\fi

%% Use microtype
\usepackage[stretch=10]{microtype}

\usepackage[dutch]{babel}

\usepackage{listings}
\lstset{
    language=C,
    basicstyle=\ttfamily\footnotesize,
    xleftmargin=1em
}

\title{THUAS Beamer Slides}
\subtitle{Jaja, het kan echt}
\author{Jesse op den Brouw}
\date{\today}


\begin{document}


%% Title page
\maketitle

\section{Test}

\begin{frame}[fragile]{THUAS Beamer Slides}
\begin{itemize}
\item Dit is de onofficiële realisatie van slides met het THUAS-thema
\item Eerst gebruik je de beamer-class: \lstinline|\documentclass{beamer}|
\item Daarna laadt je de theme met: \lstinline|\usetheme{thuas}|
\item Het werkt met \LaTeX, Xe\LaTeX en Lua\LaTeX
\item Voorlopig alleen in het Nederlands
\item De titelpagina is niet conform de regels, maar komt in de buurt
\item De navigatie-buttons werken nu nog niet
\end{itemize}
\end{frame}

\begin{frame}[fragile]{THUAS Beamer Slides}
\begin{itemize}
\item De officiële realisatie is met aspect ratio 16:9:
\begin{itemize}
\item Gebruik \lstinline|\documentclass[aspectratio=169]{beamer}|
\end{itemize}
\item Maar veel beamers werken nog met 4:3, dus:
\begin{itemize}
\item Gebruik \lstinline|\documentclass[aspectratio=43]{beamer}|
\end{itemize}
\item Er zijn nog andere formaten maar die worden niet ondersteund
\item Er zijn verschillen in de titelpagina tussen 16:9 en 4:3
\begin{itemize}
\item Dat komt o.a.\@ door het plaatsen van het plaatje op de titelpagina
\end{itemize}
\item Er zijn verschillen tussen Xe-, Lua- en pdf\LaTeX
\begin{itemize}
\item Dat komt o.a.\@ door de vorm en de grootte van de gebruikte fonts
\end{itemize}
\end{itemize}
\end{frame}

\begin{frame}{THUAS Beamer Slides}
\begin{itemize}
\item Subtitel op titelpagina wordt \emph{niet} weergegeven
\begin{itemize}
\item Deze subtitel wordt gewoon genegeerd
\end{itemize}
\item Subtitels op frames worden \emph{niet} weergegeven
\begin{itemize}
\item Deze subtitels worden gewoon genegeerd
\end{itemize}
\item De inhoud van een slide wordt \emph{niet} gecentreerd
\begin{itemize}
\item De huisstijl is zo
\end{itemize}
\end{itemize}
\end{frame}

\begin{frame}
\frametitle{Formules kunnen ook}
\begin{itemize}
\item The formules zijn:
\begin{equation*}
\left[F(x)\right]^b_a = \int_a^b x^2 + 2x + 1 \, \mathsf{d} x
\end{equation*}
\begin{equation*}
\zeta (s) = \sum_{n=1}^\infty \dfrac{1}{n^{\;\!s}}
\end{equation*}
\end{itemize}
\end{frame}

\begin{frame}
\frametitle{Voorbeeld met een itemize}
Voorbeeld met een \texttt{itemize} en gaat tot drie niveaus diep.
En nu wat tekst.
\begin{itemize}
\item item
\begin{itemize}
\item sub item
\begin{itemize}
\item sub sub item
\end{itemize}
\end{itemize}
\item two
\end{itemize}
En weer wat tekst
\end{frame}

\begin{frame}{Voorbeeld met een enumerate}
Voorbeeld met een \texttt{enumerate} en gaat tot drie niveaus diep.
\begin{enumerate}
\item een
\item twee
\begin{enumerate}
\item een
\item twee
\begin{enumerate}
\item een
\item twee
\end{enumerate}
\end{enumerate}
\item drie
\end{enumerate}
\end{frame}

%% A frame with a listing NEEDS fragile
\begin{frame}[fragile]{Een frame met code}
Een frame met daarin code met \texttt{lstlistings} moet getypeerd worden met \texttt{fragile}, anders werkt het niet.

\begin{lstlisting}
#include <stdio.h>

int main(void) {
    printf("Thuas Beamer Slides!\n");
}
\end{lstlisting}
\end{frame}

\begin{frame} 
\frametitle{There Is No Largest Prime Number} 
\begin{theorem}
There is no largest prime number.
\end{theorem} 
\begin{enumerate} 
\item<1-| alert@1> Suppose $p$ were the largest prime number. 
\item<2-> Let $q$ be the product of the first $p$ numbers. 
\item<3-> Then $q+1$ is not divisible by any of them. 
\item<1-> But $q + 1$ is greater than $1$, thus divisible by some prime
number not in the first $p$ numbers.
\end{enumerate}
\end{frame}

\end{document}
